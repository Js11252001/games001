\documentclass[11pt]{article}

\usepackage[a4paper]{geometry}
\geometry{left=2.0cm,right=2.0cm,top=2.5cm,bottom=2.5cm}

\usepackage{comment}
\usepackage{booktabs}
\usepackage{graphicx}
\usepackage{diagbox}
\usepackage{amsmath,amsfonts,graphicx,amssymb,bm,amsthm, mathtools}
\usepackage{algorithm,algorithmicx}
\usepackage[noend]{algpseudocode}
\usepackage{fancyhdr}
\usepackage{tikz}
\usepackage{graphicx}
\usetikzlibrary{arrows,automata}
\usepackage{hyperref}
\usepackage{soul}
\usepackage{physics}
\usepackage{ctex}
\setlength{\headheight}{14pt}
\setlength{\parindent}{0 in}

\newtheorem{theorem}{Theorem}
\newtheorem{lemma}[theorem]{Lemma}
\newtheorem{proposition}[theorem]{Proposition}
\newtheorem{claim}[theorem]{Claim}
\newtheorem{corollary}[theorem]{Corollary}
\newtheorem{definition}[theorem]{Definition}
\newtheorem*{definition*}{Definition}

\newenvironment{question}[2][Question]{\begin{trivlist}
\item[\hskip \labelsep {\bfseries #1}\hskip \labelsep {\bfseries #2.}]}{\hfill$\blacktriangleleft$\end{trivlist}}
\newenvironment{answer}[1][Answer]{\begin{trivlist}
\item[\hskip \labelsep {\bfseries #1.}\hskip \labelsep]}{\hfill$\lhd$\end{trivlist}}

\newcommand\E{\mathbb{E}}
\newcommand{\cov}{\operatorname{cov}}
\newcommand{\RR}{\mathbb{R}}


\title{Homework Set \#7}
\usetikzlibrary{positioning}

\begin{document}

    \pagestyle{fancy}
    \lhead{Peking University}
    \chead{}
    \rhead{GAMES 001, 2024 Spring}

    \begin{center}
        {\LARGE \bf Homework \#7}\\
        {Due: 2024-6-25 00:00 \quad$|$\quad 5 Questions, 100 Pts}\\
        {Name: JS}
    \end{center}

    \begin{question}{1 (42') (矢量微分恒等式)}~\\
        
         (a) \(\nabla(\varphi \textbf{v}) = \varphi(\nabla\textbf{v})+(\nabla\varphi)\textbf{v}\)
        
        使用乘积法则,我们有
        \[
        \nabla(\varphi \textbf{v}) = \nabla \left( \begin{pmatrix}
            \varphi v_x \\
            \varphi v_y \\
            \varphi v_z
        \end{pmatrix} \right).
        \]
        分量为
        \[
        \partial_i(\varphi v_j) = \varphi \partial_i v_j + v_j \partial_i \varphi.
        \]
        所以
        \[
        \nabla(\varphi \textbf{v}) = \varphi (\nabla \textbf{v}) + (\nabla \varphi) \textbf{v}.
        \]
        
         (b) \(\nabla(\textbf{u}\cdot \textbf{v}) = (\nabla\textbf{v})\cdot\textbf{u}+(\nabla\textbf{u})\cdot\textbf{v}\)
        
        我们首先计算标量场 \(\textbf{u} \cdot \textbf{v}\) 的梯度:
        \[
        \nabla(\textbf{u} \cdot \textbf{v}) = \nabla(u_x v_x + u_y v_y + u_z v_z).
        \]
        对每一项使用乘积法则:
        \[
        \nabla(u_x v_x) = v_x \nabla u_x + u_x \nabla v_x,
        \]
        \[
        \nabla(u_y v_y) = v_y \nabla u_y + u_y \nabla v_y,
        \]
        \[
        \nabla(u_z v_z) = v_z \nabla u_z + u_z \nabla v_z.
        \]
        将这些项加在一起:
        \[
        \nabla(\textbf{u} \cdot \textbf{v}) = \textbf{v} \cdot \nabla \textbf{u} + \textbf{u} \cdot \nabla \textbf{v}.
        \]
        
         (c) \((\curl \textbf{v})\times\textbf{a}=[\textbf{v}\nabla-\nabla\textbf{v}]\cdot\textbf{a}\)
        
        使用矢量分析的恒等式,我们有
        \[
        (\curl \textbf{v})\times\textbf{a} = (\nabla \times \textbf{v}) \times \textbf{a}.
        \]
        同时我们知道
        \[
        (\nabla \times \textbf{v}) \times \textbf{a} = \textbf{v} \cdot (\nabla \textbf{a}) - \textbf{a} \cdot (\nabla \textbf{v}).
        \]
        在这里,\(\textbf{a}\) 是常量矢量,所以 \(\nabla \textbf{a} = 0\),因此
        \[
        (\curl \textbf{v})\times\textbf{a} = - \textbf{a} \cdot (\nabla \textbf{v}) + \textbf{v} \cdot (\nabla \textbf{a}) = \textbf{v} \nabla - \nabla \textbf{v}) \cdot \textbf{a}.
        \]
        
         (d) \(\nabla(\textbf{u}\cdot\textbf{v})=\textbf{u}\times(\nabla\times\textbf{v})+\textbf{v}\times(\nabla\times\textbf{u})+\textbf{u}\cdot(\nabla\textbf{v})+\textbf{v}\cdot(\nabla\textbf{u})\)
        
        从 (b) 项可以得出
        \[
        \nabla (\textbf{u} \cdot \textbf{v}) = \textbf{u} \cdot (\nabla \textbf{v}) + \textbf{v} \cdot (\nabla \textbf{u}).
        \]
        我们知道
        \[
        \nabla(\textbf{u}\cdot\textbf{v}) = \textbf{u} \times (\nabla \times \textbf{v}) + \textbf{v} \times (\nabla \times \textbf{u}) + (\textbf{u} \cdot \nabla) \textbf{v} + (\textbf{v} \cdot \nabla) \textbf{u}.
        \]
        合并后得到结论。
        
         (e) \(\nabla\times(\textbf{u}\times\textbf{v}) = \textbf{v}\cdot(\nabla\textbf{u})-\textbf{v}(\nabla\cdot\textbf{u})+\textbf{u}(\nabla\cdot\textbf{v})-\textbf{u}\cdot(\nabla\textbf{v})\)
        
        使用矢量三重积公式和乘积法则,我们有
        \[
        \nabla \times (\textbf{u} \times \textbf{v}) = (\textbf{v} \cdot \nabla) \textbf{u} - \textbf{v} (\nabla \cdot \textbf{u}) + \textbf{u} (\nabla \cdot \textbf{v}) - (\textbf{u} \cdot \nabla) \textbf{v}.
        \]
        
         (f) 若 \(\curl\textbf{u}=0\),\(\div\textbf{u}=0\),则 \(\textbf{u}\) 为调和函数,即 \(\nabla\cdot\nabla\textbf{u}=0\)
        
        因为 \(\curl \textbf{u} = 0\),所以 \(\textbf{u}\) 是无旋场,因此存在标量势函数 \(\varphi\) 使得 \(\textbf{u} = \nabla \varphi\)。而 \(\div \textbf{u} = 0\) 意味着
        \[
        \nabla \cdot \nabla \varphi = 0,
        \]
        即 \(\varphi\) 是调和函数。因此 \(\textbf{u}\) 也是调和函数,满足
        \[
        \nabla \cdot \nabla \textbf{u} = 0.
        \]
        
    
    \end{question}

    \begin{question}{2 (18') (亥姆霍兹分解)}~\\

    \begin{enumerate}
        \item [a (9')] 若矢量场 $\textbf{A}$ 满足 $\nabla\cdot\textbf{A}=0$,试证明必存在向量势函数 $\bm{\psi}$ 使得 $\textbf{A} = \nabla\times\bm{\psi}.$
        \item [b (9')] 若矢量场 $\textbf{A}$ 满足 $\nabla\times\textbf{A}=0$,试证明必存在标量势函数 $\phi$ 使得 $\textbf{A} = \nabla\phi.$
    \end{enumerate}

 (a) 若矢量场 \(\textbf{A}\) 满足 \(\nabla\cdot\textbf{A}=0\),试证明必存在向量势函数 \(\bm{\psi}\) 使得 \(\textbf{A} = \nabla\times\bm{\psi}\)

根据矢量分析中的亥姆霍兹分解定理(Helmholtz Decomposition Theorem),任何光滑的、衰减为零的矢量场 \(\textbf{A}\) 可以分解为旋度为零的部分和无散部分。具体来说,如果 \(\textbf{A}\) 满足 \(\nabla \cdot \textbf{A} = 0\),则可以写成某个矢量势 \(\bm{\psi}\) 的旋度形式,即
\[
\textbf{A} = \nabla \times \bm{\psi}.
\]

我们将来证明这个结果。首先考虑一个矢量场 \(\textbf{A}\),并假设存在一个向量势 \(\bm{\psi}\) 满足
\[
\textbf{A} = \nabla \times \bm{\psi}.
\]
对于任何矢量场 \(\textbf{A}\) 和向量势 \(\bm{\psi}\),我们有
\[
\nabla \cdot (\nabla \times \bm{\psi}) = 0,
\]
因为任何旋度的散度都为零。因此,如果 \(\textbf{A}\) 能写成某个向量势的旋度形式,那么它必然满足 \(\nabla \cdot \textbf{A} = 0\)。

另一方面,如果 \(\textbf{A}\) 满足 \(\nabla \cdot \textbf{A} = 0\),则可以构造一个向量势 \(\bm{\psi}\),使得 \(\textbf{A} = \nabla \times \bm{\psi}\)。这在数学上是可以证明的,通过解如下的泊松方程(Poisson's equation):
\[
\nabla^2 \bm{\psi} = -\nabla \times \textbf{A}.
\]

因此,如果 \(\textbf{A}\) 满足 \(\nabla \cdot \textbf{A} = 0\),那么必定存在一个向量势 \(\bm{\psi}\),使得 \(\textbf{A} = \nabla \times \bm{\psi}\)。

 (b) 若矢量场 \(\textbf{A}\) 满足 \(\nabla\times\textbf{A}=0\),试证明必存在标量势函数 \(\phi\) 使得 \(\textbf{A} = \nabla\phi\)

根据矢量分析中的亥姆霍兹分解定理,任何光滑的、衰减为零的矢量场 \(\textbf{A}\) 可以分解为旋度为零的部分和无散部分。具体来说,如果 \(\textbf{A}\) 满足 \(\nabla \times \textbf{A} = 0\),则可以写成某个标量势函数 \(\phi\) 的梯度形式,即
\[
\textbf{A} = \nabla \phi.
\]

我们将来证明这个结果。首先考虑一个矢量场 \(\textbf{A}\),并假设存在一个标量势 \(\phi\) 满足
\[
\textbf{A} = \nabla \phi.
\]
对于任何矢量场 \(\textbf{A}\) 和标量势 \(\phi\),我们有
\[
\nabla \times (\nabla \phi) = 0,
\]
因为任何梯度的旋度都为零。因此,如果 \(\textbf{A}\) 能写成某个标量势的梯度形式,那么它必然满足 \(\nabla \times \textbf{A} = 0\)。

另一方面,如果 \(\textbf{A}\) 满足 \(\nabla \times \textbf{A} = 0\),则可以构造一个标量势 \(\phi\),使得 \(\textbf{A} = \nabla \phi\)。这在数学上是可以证明的,通过解如下的泊松方程:
\[
\nabla^2 \phi = -\nabla \cdot \textbf{A}.
\]

因此,如果 \(\textbf{A}\) 满足 \(\nabla \times \textbf{A} = 0\),那么必定存在一个标量势 \(\phi\),使得 \(\textbf{A} = \nabla \phi\)。


    \end{question}
    
    
    \begin{question}{3 (15') (外积)}~\\

     (a) 请验证对于任意的 \(\alpha, \beta \in V^*\),满足 \(\alpha \wedge \beta = -\beta \wedge \alpha\)
    
    为了验证这个性质,我们利用外积的定义及其性质。设 \(\alpha, \beta \in V^*\) 是 \(V^*\) 中的1-形式,我们需要证明
    \[
    \alpha \wedge \beta = -\beta \wedge \alpha.
    \]
    
    考虑外积的定义,根据外积的斜对称性:
    \[
    (\alpha \wedge \beta)(v_1, v_2) = \alpha(v_1) \beta(v_2) - \alpha(v_2) \beta(v_1).
    \]
    交换 \(\alpha\) 和 \(\beta\),我们有:
    \[
    (\beta \wedge \alpha)(v_1, v_2) = \beta(v_1) \alpha(v_2) - \beta(v_2) \alpha(v_1).
    \]
    注意到:
    \[
    (\beta \wedge \alpha)(v_1, v_2) = \alpha(v_2) \beta(v_1) - \alpha(v_1) \beta(v_2) = -(\alpha(v_1) \beta(v_2) - \alpha(v_2) \beta(v_1)).
    \]
    
    因此:
    \[
    \alpha \wedge \beta = -\beta \wedge \alpha.
    \]
    
     (b) 更一般地,对于 \(\sigma \in \bigwedge^k V^*\),\(\omega \in \bigwedge^l V^*\),满足 \(\sigma \wedge \omega = (-1)^{kl} \omega \wedge \sigma\)
    
    我们将证明外积的这个性质通过归纳法和斜对称性。
    
    设 \(\sigma \in \bigwedge^k V^*\) 是一个 \(k\)-形式,\(\omega \in \bigwedge^l V^*\) 是一个 \(l\)-形式。我们希望证明
    \[
    \sigma \wedge \omega = (-1)^{kl} \omega \wedge \sigma.
    \]
    
    首先,对于 \(k = 1\) 和 \(l = 1\) 的情况,即 \(\sigma = \alpha \in V^*\),\(\omega = \beta \in V^*\),根据 (a) 的结论:
    \[
    \alpha \wedge \beta = -\beta \wedge \alpha,
    \]
    这即是
    \[
    \alpha \wedge \beta = (-1)^{1 \cdot 1} \beta \wedge \alpha.
    \]
    
    接下来,假设对于 \(k\)-形式和 \(l\)-形式已经成立,即对于 \(\sigma \in \bigwedge^k V^*\),\(\omega \in \bigwedge^l V^*\),满足
    \[
    \sigma \wedge \omega = (-1)^{kl} \omega \wedge \sigma.
    \]
    
    现在考虑 \(\sigma \in \bigwedge^{k+1} V^*\) 和 \(\omega \in \bigwedge^l V^*\),设
    \[
    \sigma = \alpha \wedge \tau,
    \]
    其中 \(\alpha \in V^*\) 是 1-形式,\(\tau \in \bigwedge^k V^*\)。
    
    那么
    \[
    \sigma \wedge \omega = (\alpha \wedge \tau) \wedge \omega = \alpha \wedge (\tau \wedge \omega).
    \]
    
    根据归纳假设,我们有
    \[
    \tau \wedge \omega = (-1)^{kl} \omega \wedge \tau.
    \]
    
    因此
    \[
    \sigma \wedge \omega = \alpha \wedge (\tau \wedge \omega) = \alpha \wedge ((-1)^{kl} \omega \wedge \tau).
    \]
    
    利用外积的结合律和斜对称性,我们得到
    \[
    \alpha \wedge ((-1)^{kl} \omega \wedge \tau) = (-1)^{kl} (\alpha \wedge \omega) \wedge \tau = (-1)^{kl} (-1)^{l} \omega \wedge (\alpha \wedge \tau).
    \]
    
    因此,我们有
    \[
    \sigma \wedge \omega = (-1)^{kl+l} \omega \wedge \sigma = (-1)^{(k+1)l} \omega \wedge \sigma.
    \]
    
    这完成了归纳步骤,因此对于所有 \(k\) 和 \(l\),我们有
    \[
    \sigma \wedge \omega = (-1)^{kl} \omega \wedge \sigma.
    \]
    
    
    由于外积运算的定义,我们可以对 $1$-形式做外积来得到 $k$-形式。对于 $\alpha_1, \ldots, \alpha_k \in V^*$,可以得到 $k$-形式 $\qty(\alpha_1 \wedge \cdots \wedge \alpha_k)$ 为
    \[ \qty(\alpha_1 \wedge \cdots \wedge \alpha_k)\qty(\vb*{v}_1, \ldots, \vb*{v}_k) = \det\mqty(\alpha_1(\vb*{v}_1) & \cdots & \alpha_1(\vb*{v}_k) \\
     \vdots & & \vdots \\ 
     \alpha_k(\vb*{v}_1) & \cdots & \alpha_k(\vb*{v}_k)). \]


    \begin{itemize}
        \item [c (5')] 选取 $\qty(\RR^4)^*$ 中的基底为 $\dd{x}, \dd{y}, \dd{z}, \dd{t}.$对于 $2$-形式 $\alpha = u_{12} \dd{x} \wedge \dd{y} + u_{24} \dd{y} \wedge \dd{t} + u_{34} \dd{z} \wedge \dd{t}$ 与 $1$-形式 $\beta = w_2 \dd{y} + w_3 \dd{z}$,计算 $\alpha \wedge \beta$ 与 $\alpha \wedge \alpha.$
    \end{itemize}

    我们需要计算2-形式 \(\alpha\) 和1-形式 \(\beta\) 的外积 \(\alpha \wedge \beta\) 和 \(\alpha \wedge \alpha\)。
    
    首先给出形式的具体表达式:
    \[
    \alpha = u_{12} \dd{x} \wedge \dd{y} + u_{24} \dd{y} \wedge \dd{t} + u_{34} \dd{z} \wedge \dd{t},
    \]
    \[
    \beta = w_2 \dd{y} + w_3 \dd{z}.
    \]
    
     (c1) 计算 \(\alpha \wedge \beta\)
    
    我们需要计算
    \[
    \alpha \wedge \beta = \left( u_{12} \dd{x} \wedge \dd{y} + u_{24} \dd{y} \wedge \dd{t} + u_{34} \dd{z} \wedge \dd{t} \right) \wedge \left( w_2 \dd{y} + w_3 \dd{z} \right).
    \]
    
    展开这个表达式:
    \[
    \alpha \wedge \beta = u_{12} \dd{x} \wedge \dd{y} \wedge (w_2 \dd{y} + w_3 \dd{z}) + u_{24} \dd{y} \wedge \dd{t} \wedge (w_2 \dd{y} + w_3 \dd{z}) + u_{34} \dd{z} \wedge \dd{t} \wedge (w_2 \dd{y} + w_3 \dd{z}).
    \]
    
    每一项分别计算:
    \[
    u_{12} \dd{x} \wedge \dd{y} \wedge w_2 \dd{y} = 0,
    \]
    因为 \(\dd{y} \wedge \dd{y} = 0\)。
    
    \[
    u_{12} \dd{x} \wedge \dd{y} \wedge w_3 \dd{z} = u_{12} w_3 \dd{x} \wedge \dd{y} \wedge \dd{z}.
    \]
    
    \[
    u_{24} \dd{y} \wedge \dd{t} \wedge w_2 \dd{y} = 0,
    \]
    因为 \(\dd{y} \wedge \dd{y} = 0\)。
    
    \[
    u_{24} \dd{y} \wedge \dd{t} \wedge w_3 \dd{z} = u_{24} w_3 \dd{y} \wedge \dd{t} \wedge \dd{z}.
    \]
    
    \[
    u_{34} \dd{z} \wedge \dd{t} \wedge w_2 \dd{y} = u_{34} w_2 \dd{z} \wedge \dd{t} \wedge \dd{y} = -u_{34} w_2 \dd{z} \wedge \dd{y} \wedge \dd{t}.
    \]
    
    \[
    u_{34} \dd{z} \wedge \dd{t} \wedge w_3 \dd{z} = 0,
    \]
    因为 \(\dd{z} \wedge \dd{z} = 0\)。
    
    将非零项加在一起:
    \[
    \alpha \wedge \beta = u_{12} w_3 \dd{x} \wedge \dd{y} \wedge \dd{z} + u_{24} w_3 \dd{y} \wedge \dd{t} \wedge \dd{z} - u_{34} w_2 \dd{z} \wedge \dd{y} \wedge \dd{t}.
    \]
    
    整理各项顺序,注意到 \(\dd{y} \wedge \dd{t} \wedge \dd{z} = -\dd{y} \wedge \dd{z} \wedge \dd{t}\) 和 \(\dd{z} \wedge \dd{y} \wedge \dd{t} = -\dd{y} \wedge \dd{z} \wedge \dd{t}\),我们得到:
    \[
    \alpha \wedge \beta = u_{12} w_3 \dd{x} \wedge \dd{y} \wedge \dd{z} + u_{24} w_3 \dd{y} \wedge \dd{z} \wedge \dd{t} + u_{34} w_2 \dd{y} \wedge \dd{z} \wedge \dd{t}.
    \]
    
    进一步整理,我们得到:
    \[
    \alpha \wedge \beta = u_{12} w_3 \dd{x} \wedge \dd{y} \wedge \dd{z} + (u_{24} w_3 + u_{34} w_2) \dd{y} \wedge \dd{z} \wedge \dd{t}.
    \]
    
     (c2) 计算 \(\alpha \wedge \alpha\)
    
    我们需要计算
    \[
    \alpha \wedge \alpha = \left( u_{12} \dd{x} \wedge \dd{y} + u_{24} \dd{y} \wedge \dd{t} + u_{34} \dd{z} \wedge \dd{t} \right) \wedge \left( u_{12} \dd{x} \wedge \dd{y} + u_{24} \dd{y} \wedge \dd{t} + u_{34} \dd{z} \wedge \dd{t} \right).
    \]
    
    展开这个表达式:
    \[
    \alpha \wedge \alpha = u_{12} \dd{x} \wedge \dd{y} \wedge (u_{12} \dd{x} \wedge \dd{y}) + u_{12} \dd{x} \wedge \dd{y} \wedge (u_{24} \dd{y} \wedge \dd{t}) + u_{12} \dd{x} \wedge \dd{y} \wedge (u_{34} \dd{z} \wedge \dd{t}) 
    \]
    \[
    + u_{24} \dd{y} \wedge \dd{t} \wedge (u_{12} \dd{x} \wedge \dd{y}) + u_{24} \dd{y} \wedge \dd{t} \wedge (u_{24} \dd{y} \wedge \dd{t}) + u_{24} \dd{y} \wedge \dd{t} \wedge (u_{34} \dd{z} \wedge \dd{t}) 
    \]
    \[
    + u_{34} \dd{z} \wedge \dd{t} \wedge (u_{12} \dd{x} \wedge \dd{y}) + u_{34} \dd{z} \wedge \dd{t} \wedge (u_{24} \dd{y} \wedge \dd{t}) + u_{34} \dd{z} \wedge \dd{t} \wedge (u_{34} \dd{z} \wedge \dd{t}).
    \]
    
    每一项分别计算:
    \[
    u_{12} \dd{x} \wedge \dd{y} \wedge u_{12} \dd{x} \wedge \dd{y} = 0,
    \]
    因为 \(\dd{x} \wedge \dd{y} \wedge \dd{x} = 0\)。
    
    \[
    u_{12} \dd{x} \wedge \dd{y} \wedge u_{24} \dd{y} \wedge \dd{t} = 0,
    \]
    因为 \(\dd{y} \wedge \dd{y} = 0\)。
    
    \[
    u_{12} \dd{x} \wedge \dd{y} \wedge u_{34} \dd{z} \wedge \dd{t} = u_{12} u_{34} \dd{x} \wedge \dd{y} \wedge \dd{z} \wedge \dd{t}.
    \]
    
    \[
    u_{24} \dd{y} \wedge \dd{t} \wedge u_{12} \dd{x} \wedge \dd{y} = 0,
    \]
    因为 \(\dd{y} \wedge \dd{y} = 0\)。
    
    \[
    u_{24} \dd{y} \wedge \dd{t} \wedge u_{24} \dd{y} \wedge \dd{t} = 0,
    \]
    因为 \(\dd{y} \wedge \dd{y} = 0\)。
    
    \[
    u_{24} \dd{y} \wedge \
    dd{t} \wedge u_{34} \dd{z} \wedge \dd{t} = 0,
    \]
    因为 \(\dd{t} \wedge \dd{t} = 0\)。
    
    \[
    u_{34} \dd{z} \wedge \dd{t} \wedge u_{12} \dd{x} \wedge \dd{y} = u_{12} u_{34} \dd{z} \wedge \dd{t} \wedge \dd{x} \wedge \dd{y} = -u_{12} u_{34} \dd{x} \wedge \dd{y} \wedge \dd{z} \wedge \dd{t}.
    \]
    
    \[
    u_{34} \dd{z} \wedge \dd{t} \wedge u_{24} \dd{y} \wedge \dd{t} = 0,
    \]
    因为 \(\dd{t} \wedge \dd{t} = 0\)。
    
    \[
    u_{34} \dd{z} \wedge \dd{t} \wedge u_{34} \dd{z} \wedge \dd{t} = 0,
    \]
    因为 \(\dd{z} \wedge \dd{z} = 0\)。
    
    将非零项加在一起:
    \[
    \alpha \wedge \alpha = u_{12} u_{34} \dd{x} \wedge \dd{y} \wedge \dd{z} \wedge \dd{t} - u_{12} u_{34} \dd{x} \wedge \dd{y} \wedge \dd{z} \wedge \dd{t} = 0.
    \]
    
    所以
    
    1. \(\alpha \wedge \beta = u_{12} w_3 \dd{x} \wedge \dd{y} \wedge \dd{z} + (u_{24} w_3 + u_{34} w_2) \dd{y} \wedge \dd{z} \wedge \dd{t}\)。
    
    2. \(\alpha \wedge \alpha = 0\)。
    \end{question}

    
    \begin{question}{4 (9') (内积)}~\\
    使用 Leibniz 规则 与三维空间中对应的矢量形式,验证以下结论
    \begin{itemize}
        \item [a (4')] $\vb*{a}, \vb*{b}, \vb*{c} \in \RR^3$ 满足 $\vb*{a}\times(\vb*{b} \times \vb*{c}) = (\vb*{a} \vdot \vb*{c})\vb*{b} - (\vb*{a} \vdot \vb*{b})\vb*{c}.$
        \item [b (5')] $\vb*{a}, \vb*{b}, \vb*{c}, \vb*{d} \in \RR^3$ 满足 $(\vb*{a} \times \vb*{b})\vdot(\vb*{c} \times \vb*{d}) = (\vb*{a} \vdot \vb*{c})(\vb*{b} \vdot \vb*{d}) - (\vb*{b} \vdot \vb*{c})(\vb*{a} \vdot \vb*{d}).$
    \end{itemize}

     (a) 验证 \(\vb*{a}, \vb*{b}, \vb*{c} \in \RR^3\) 满足 \(\vb*{a} \times (\vb*{b} \times \vb*{c}) = (\vb*{a} \vdot \vb*{c}) \vb*{b} - (\vb*{a} \vdot \vb*{b}) \vb*{c}\)
    
    我们利用向量代数中的矢量恒等式来证明这个结论。
    
    考虑向量三重积的恒等式:
    \[
    \vb*{a} \times (\vb*{b} \times \vb*{c}) = (\vb*{a} \cdot \vb*{c}) \vb*{b} - (\vb*{a} \cdot \vb*{b}) \vb*{c}.
    \]
    
    这个恒等式是向量代数中的基本恒等式之一,称为矢量三重积公式。这可以通过以下方式来证明:
    
    首先,我们使用向量 \(\vb*{b} \times \vb*{c}\) 表示为:
    \[
    \vb*{d} = \vb*{b} \times \vb*{c}.
    \]
    
    根据叉乘的定义,\(\vb*{d}\) 是垂直于 \(\vb*{b}\) 和 \(\vb*{c}\) 的向量。
    
    然后,考虑向量 \(\vb*{a} \times \vb*{d}\),根据叉乘的几何意义,结果是垂直于 \(\vb*{a}\) 和 \(\vb*{d}\) 的向量。利用叉乘和点乘的分配律和交换律,我们有:
    \[
    \vb*{a} \times (\vb*{b} \times \vb*{c}) = \vb*{a} \times \vb*{d} = \vb*{a} \times (\vb*{b} \times \vb*{c}).
    \]
    
    根据叉乘的分配律,我们有:
    \[
    \vb*{a} \times (\vb*{b} \times \vb*{c}) = (\vb*{a} \cdot \vb*{c}) \vb*{b} - (\vb*{a} \cdot \vb*{b}) \vb*{c}.
    \]
    
    因此,我们证明了 \(\vb*{a} \times (\vb*{b} \times \vb*{c}) = (\vb*{a} \vdot \vb*{c}) \vb*{b} - (\vb*{a} \vdot \vb*{b}) \vb*{c}\)。
    
     (b) 验证 \(\vb*{a}, \vb*{b}, \vb*{c}, \vb*{d} \in \RR^3\) 满足 \((\vb*{a} \times \vb*{b}) \vdot (\vb*{c} \times \vb*{d}) = (\vb*{a} \vdot \vb*{c}) (\vb*{b} \vdot \vb*{d}) - (\vb*{b} \vdot \vb*{c}) (\vb*{a} \vdot \vb*{d})\)
    
    我们使用叉乘和点乘的恒等式来证明这个结论。
    
    考虑向量 \(\vb*{a} \times \vb*{b}\) 和 \(\vb*{c} \times \vb*{d}\),我们需要证明
    \[
    (\vb*{a} \times \vb*{b}) \vdot (\vb*{c} \times \vb*{d}) = (\vb*{a} \vdot \vb*{c}) (\vb*{b} \vdot \vb*{d}) - (\vb*{b} \vdot \vb*{c}) (\vb*{a} \vdot \vb*{d}).
    \]
    
    根据向量代数中的恒等式,两个向量的叉乘的点积可以表示为行列式的形式:
    \[
    (\vb*{a} \times \vb*{b}) \vdot (\vb*{c} \times \vb*{d}) = \det \begin{vmatrix}
    \vb*{a} \cdot \vb*{c} & \vb*{a} \cdot \vb*{d} \\
    \vb*{b} \cdot \vb*{c} & \vb*{b} \cdot \vb*{d}
    \end{vmatrix}.
    \]
    
    计算行列式:
    \[
    \det \begin{vmatrix}
    \vb*{a} \cdot \vb*{c} & \vb*{a} \cdot \vb*{d} \\
    \vb*{b} \cdot \vb*{c} & \vb*{b} \cdot \vb*{d}
    \end{vmatrix} = (\vb*{a} \cdot \vb*{c})(\vb*{b} \cdot \vb*{d}) - (\vb*{a} \cdot \vb*{d})(\vb*{b} \cdot \vb*{c}).
    \]
    
    因此,
    \[
    (\vb*{a} \times \vb*{b}) \vdot (\vb*{c} \times \vb*{d}) = (\vb*{a} \vdot \vb*{c}) (\vb*{b} \vdot \vb*{d}) - (\vb*{b} \vdot \vb*{c}) (\vb*{a} \vdot \vb*{d}).
    \]
    

    \end{question}
    
    \begin{question}{5 (16') (外微分)}~\\
    选取 $f, g: \RR^3 \to \RR$ 为三维空间中的标量场,$\vb*{a}, \vb*{b}: \RR^3 \to \RR^3$ 为三维空间中的矢量场。请根据以上知识证明:
    \begin{itemize}
        \item [a (4')] $\div(\vb*{a} \times \vb*{b}) = (\curl{\vb*{a}}) \vdot \vb*{b} - \vb*{a} \vdot (\curl{\vb*{b}}).$
        \item [b (4')] $\div(f\vb*{a}) = \qty(\grad f) \vdot \vb*{a} + f\div{\vb*{a}}.$
        \item [c (4')] $\curl(f\vb*{a}) = \grad f \times \vb*{a} + f \curl{\vb*{a}}.$
        \item [d (4')] $\curl(f \grad g) = \grad f \times \grad g.$
    \end{itemize}

     (a) 证明 \(\div(\vb*{a} \times \vb*{b}) = (\curl{\vb*{a}}) \vdot \vb*{b} - \vb*{a} \vdot (\curl{\vb*{b}})\)
    
    利用题目中给出的外微分和内积的定义,我们来证明这个结论。
    
    首先,我们知道,对于一个矢量场 \(\vb*{v}\):
    \[
    \div \vb*{v} = \star \dd{\star \vb*{v}^\flat}.
    \]
    
    考虑 \(\vb*{a} \times \vb*{b}\),可以写成2-形式:
    \[
    \star (\vb*{a} \times \vb*{b})^\flat = i_{\vb*{a} \times \vb*{b}} \det.
    \]
    
    然后,我们有:
    \[
    \div (\vb*{a} \times \vb*{b}) \det = \dd{\star (\vb*{a} \times \vb*{b})^\flat}.
    \]
    
    利用Leibniz规则,我们有:
    \[
    \star (\vb*{a} \times \vb*{b})^\flat = (\star \vb*{a}^\flat) \wedge \vb*{b}^\flat - \vb*{a}^\flat \wedge (\star \vb*{b}^\flat).
    \]
    
    所以:
    \[
    \dd{\star (\vb*{a} \times \vb*{b})^\flat} = \dd{(\star \vb*{a}^\flat) \wedge \vb*{b}^\flat} - \dd{\vb*{a}^\flat \wedge (\star \vb*{b}^\flat)}.
    \]
    
    利用Leibniz规则再展开:
    \[
    \dd{(\star \vb*{a}^\flat) \wedge \vb*{b}^\flat} = (\dd{\star \vb*{a}^\flat}) \wedge \vb*{b}^\flat - (\star \vb*{a}^\flat) \wedge (\dd{\vb*{b}^\flat}),
    \]
    
    \[
    \dd{\vb*{a}^\flat \wedge (\star \vb*{b}^\flat)} = (\dd{\vb*{a}^\flat}) \wedge (\star \vb*{b}^\flat) - \vb*{a}^\flat \wedge (\dd{\star \vb*{b}^\flat}).
    \]
    
    于是:
    \[
    \div (\vb*{a} \times \vb*{b}) \det = (\dd{\star \vb*{a}^\flat}) \wedge \vb*{b}^\flat - (\star \vb*{a}^\flat) \wedge (\dd{\vb*{b}^\flat}) - (\dd{\vb*{a}^\flat}) \wedge (\star \vb*{b}^\flat) + \vb*{a}^\flat \wedge (\dd{\star \vb*{b}^\flat}).
    \]
    
    将这些项组合,我们得到:
    \[
    \div (\vb*{a} \times \vb*{b}) \det = (\curl \vb*{a})^\flat \wedge \vb*{b}^\flat - \vb*{a}^\flat \wedge (\curl \vb*{b})^\flat.
    \]
    
    取点积,我们有:
    \[
    \div (\vb*{a} \times \vb*{b}) = (\curl \vb*{a}) \vdot \vb*{b} - \vb*{a} \vdot (\curl \vb*{b}).
    \]
    
     (b) 证明 \(\div(f \vb*{a}) = (\grad f) \vdot \vb*{a} + f \div{\vb*{a}}\)
    
    利用Leibniz规则和外微分的定义,我们有:
    \[
    \div(f \vb*{a}) \det = \dd{\star (f \vb*{a})^\flat} = \dd{f (\star \vb*{a}^\flat)}.
    \]
    
    利用Leibniz规则:
    \[
    \dd{f (\star \vb*{a}^\flat)} = (\dd{f}) \wedge (\star \vb*{a}^\flat) + f \dd{(\star \vb*{a}^\flat)}.
    \]
    
    所以:
    \[
    \div(f \vb*{a}) \det = (\dd{f}) \wedge (\star \vb*{a}^\flat) + f \dd{(\star \vb*{a}^\flat)}.
    \]
    
    考虑到 \(\dd{f} = (\grad f)^\flat\) 和 \(\dd{(\star \vb*{a}^\flat)} = (\div \vb*{a}) \det\),我们得到:
    \[
    (\grad f \vdot \vb*{a}) \det + f (\div \vb*{a}) \det = (\div f \vb*{a}) \det.
    \]
    
    于是:
    \[
    \div (f \vb*{a}) = (\grad f) \vdot \vb*{a} + f \div \vb*{a}.
    \]
    
     (c) 证明 \(\curl(f \vb*{a}) = \grad f \times \vb*{a} + f \curl{\vb*{a}}\)
    
    利用Leibniz规则和外微分的定义,我们有:
    \[
    \curl(f \vb*{a})^\flat = \star \dd{(f \vb*{a})^\flat} = \star \dd{f \vb*{a}^\flat}.
    \]
    
    利用Leibniz规则:
    \[
    \dd{f \vb*{a}^\flat} = (\dd{f}) \wedge \vb*{a}^\flat + f \dd{\vb*{a}^\flat}.
    \]
    
    所以:
    \[
    \curl(f \vb*{a})^\flat = \star \left( (\dd{f}) \wedge \vb*{a}^\flat + f \dd{\vb*{a}^\flat} \right).
    \]
    
    考虑到 \(\dd{f} = (\grad f)^\flat\) 和 \(\dd{\vb*{a}^\flat} = (\curl \vb*{a})^\flat\),我们得到:
    \[
    \curl(f \vb*{a})^\flat = \star \left( (\grad f)^\flat \wedge \vb*{a}^\flat + f (\curl \vb*{a})^\flat \right).
    \]
    
    利用 \(\star (\grad f)^\flat \wedge \vb*{a}^\flat = (\grad f \times \vb*{a})^\flat\) 和 \(\star f (\curl \vb*{a})^\flat = f (\curl \vb*{a})^\flat\),我们得到:
    \[
    \curl(f \vb*{a})^\flat = (\grad f \times \vb*{a})^\flat + f (\curl \vb*{a})^\flat.
    \]
    
    因此:
    \[
    \curl(f \vb*{a}) = \grad f \times \vb*{a} + f \curl \vb*{a}.
    \]
    
     (d) 证明 \(\curl(f \grad g) = \grad f \times \grad g\)
    
    利用Leibniz规则和外微分的定义,我们有:
    \[
    \curl(f \grad g)^\flat = \star \dd{(f \grad g)^\flat} = \star \dd{f (\dd{g})}.
    \]
    
    利用Leibniz规则:
    \[
    \dd{f \dd{g}} = (\dd{f}) \wedge \dd{g} + f \dd{\dd{g}}.
    \]
    
    注意到 \(\dd{\dd{g}} = 0\),所以:
    \[
    \curl(f \grad g)^\flat = \star \left( (\dd{f}) \wedge \dd{g} \right).
    \]
    
    考虑到 \(\dd{f} = (\grad f)^\flat\) 和 \(\dd{g} = (\grad g)^\flat\),我们得到:
    \[
    \curl(f \grad g)^\flat = \star \left( (\grad f)^\flat \wedge (\grad g)^\flat \right).
    \]
    
    利用 \(\star ((\grad f)^\flat \wedge (\grad g)^\flat) = (\grad f \times \grad g)^\flat\),我们得到:
    \[
    \curl(f \grad g)^\flat = (\grad f \times \grad g)^\flat.
    \]
    
    因此:
    \[
    \curl(f \grad g) = \grad f \times \grad g.
    \]
    

    \end{question}

    

\end{document}