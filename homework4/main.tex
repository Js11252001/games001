\documentclass[11pt]{article}

\usepackage[a4paper]{geometry}
\geometry{left=2.0cm,right=2.0cm,top=2.5cm,bottom=2.5cm}

\usepackage{comment}
\usepackage{booktabs}
\usepackage{graphicx}
\usepackage{diagbox}
\usepackage{amsmath,amsfonts,graphicx,amssymb,bm,amsthm}
%\usepackage{algorithm,algorithmicx}
\usepackage[ruled]{algorithm2e}
\usepackage[noend]{algpseudocode}
\usepackage{fancyhdr}
\usepackage{tikz}
\usepackage{graphicx}
\usetikzlibrary{arrows,automata}
\usepackage{hyperref}
\usepackage{physics}
\usepackage{ctex}

\makeatletter
\newcommand{\rmnum}[1]{\romannumeral #1}
\makeatother

\setlength{\headheight}{14pt}
\setlength{\parindent}{0 in}

\newtheorem{theorem}{Theorem}
\newtheorem{lemma}[theorem]{Lemma}
\newtheorem{proposition}[theorem]{Proposition}
\newtheorem{claim}[theorem]{Claim}
\newtheorem{corollary}[theorem]{Corollary}
\newtheorem{definition}[theorem]{Definition}

\newenvironment{question}[2][Question]{\begin{trivlist}
\item[\hskip \labelsep {\bfseries #1}\hskip \labelsep {\bfseries #2.}]}{\hfill$\blacktriangleleft$\end{trivlist}}
\newenvironment{answer}[1][Answer]{\begin{trivlist}
\item[\hskip \labelsep {\bfseries #1.}\hskip \labelsep]}{\hfill$\lhd$\end{trivlist}}

\newcommand\E{\mathbb{E}}

\def\onedot{$\mathsurround0pt\ldotp$}
\def\cddot{% two dots stacked vertically
  \mathbin{\vcenter{\baselineskip.67ex
    \hbox{\onedot}\hbox{\onedot}}%
  }}%
\def\cdddot#1{% three dots 
  \mathbin{\vcenter{\baselineskip.67ex
    \hbox{\onedot}\hbox{\onedot}\hbox{\onedot}%
  }}%
}


\title{Homework Set \#4}
\usetikzlibrary{positioning}

\begin{document}

    \pagestyle{fancy}
    \lhead{}
    \chead{}
    \rhead{GAMES 001, 2024 Spring}

    \begin{center}
        {\LARGE \bf Homework \#4}\\
        {Due: 2024-5-14 00:00 \quad$|$\quad 1 Questions, 100 Pts}\\
        {Name: JS}
    \end{center}

    \begin{question}{1 (100') (插值)}~
    假设采样点分布在$x_i=0, 1, 2, \cdots$的整点上,$x_i$对应的采样值是$y_i$,试求三阶厄米插值(课件第16页)在Catmull-Rom样条假设(课件第17页)下的基函数,最终应该得到的是如课件第25页所示那样的函数图形。
    
    提示:你可以先代入Catmull-Rom样条假设的条件将插值曲线$f(x)$写为仅包含采样值$y_i$而不包含$m_i$的形式,然后思考如果将这一关系转化为$f(x)=\sum B_i y_i$。注意最终得到的基函数是一个分段函数。
    \end{question}

    
    在给定的区间 \([x_i, x_{i+1}]\) 上,三阶厄米插值多项式 \(f(x)\) 可以表示为:
    
    \[ f(t) = h_{00}(t) y_i + h_{10}(t) m_i (x_{i+1} - x_i ) + h_{01}(t) y_{i+1} + h_{11}(t) m_{i+1} (x_{i+1} - x_i ) \]
    
    其中 \(t\) 是归一化的参数,\( t = \frac{x - x_i}{x_{i+1} - x_i} \)。
    
    三阶多项式基函数 \(h_{jk}(t)\) 定义如下:
    
     \(h_{00}(t) = 2t^3 - 3t^2 + 1\)

     \(h_{10}(t) = t^3 - 2t^2 + t\)

     \(h_{01}(t) = -2t^3 + 3t^2\)

     \(h_{11}(t) = t^3 - t^2\)
    
    在Catmull-Rom样条中,切线 \(m_i\) 由相邻点的函数值确定:
    
    \[ m_i = \frac{y_{i+1} - y_{i-1}}{2} \]
    \[ m_{i+1} = \frac{y_{i+2} - y_i}{2} \]
    
    将上述表达式替换到三阶厄米多项式中,得到:
    
    \[ f(t) = h_{00}(t) y_i + h_{10}(t) \frac{y_{i+1} - y_{i-1}}{2}(x_{i+1} - x_i ) + h_{01}(t) y_{i+1} + h_{11}(t) \frac{y_{i+2} - y_i}{2}(x_{i+1} - x_i ) \]
    
    因为$x_{i+1} - x_i = 1$
    
    我们的目标是将 \(f(t)\) 表达为 \(f(x) = \sum_j B_j(x) y_j\) 的形式。我们需要展开 \(f(t)\) 并整理系数:
    
    \[
    f(t) = \left( h_{00}(t) - \frac{1}{2} h_{10}(t) + \frac{1}{2} h_{11}(t) \right) y_i + \left( \frac{1}{2} h_{10}(t) + h_{01}(t) \right) y_{i+1} + \frac{1}{2} h_{11}(t) y_{i+2} - \frac{1}{2} h_{10}(t) y_{i-1}
    \]
    
    所以,我们得到每个 \(y_j\) 的基函数 \(B_j(x)\) 如下:
    
    - \( B_{i-1}(x) = -\frac{1}{2} h_{10}(t) \)

    - \( B_i(x) = h_{00}(t) - \frac{1}{2} h_{10}(t) + \frac{1}{2} h_{11}(t) \)

    - \( B_{i+1}(x) = \frac{1}{2} h_{10}(t) + h_{01}(t) \)
    
    - \( B_{i+2}(x) = \frac{1}{2} h_{11}(t) \)
    
    这些基函数定义了在每个区间 \([x_i, x_{i+1}]\) 上的插值行为,并且保证了插值函数的平滑连续性。基函数 \(B_j(x)\) 是分段定义的,并且可以直接用于构造整个曲线。这些基函数也可以用来绘制函数图形,展示Catmull-Rom样条的形状。
\end{document}